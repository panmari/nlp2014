\documentclass[11pt]{article}
\usepackage{amsmath}

%Gummi|065|=)
\title{\textbf{Answers Exercise 9 - Evaluating Classifications}}
\author{Stefan Moser}
\date{\today}
\begin{document}

\maketitle

\section*{Exercise 1 - Authorship attribution}
\setcounter{section}{1}
\subsection{Burrows Delta}
For computing burrows delta, we need the 
frequencies of the $n$ most common words in a default text in the same language. I will
assume that T1 to T5 fulfill this criteria (See Table \ref{table:count}). 
\begin{table}[h]
\center
\begin{tabular}{|l|l|l|l|l|}
\hline
	 & A1 & A2 & A3 & Q\\
\hline
	T1 & 20 & 21 & 3 & 25\\
\hline
	T2 & 36 & 100 & 23 & 40\\
\hline
	T3 & 90 & 100 & 12 & 45\\
\hline
	T4 & 75 & 3 & 67 & 10\\
\hline
	T5 & 12 & 57 & 34 & 20\\
\hline
	size & 233 & 281 & 139 & 140\\
\hline
\end{tabular}
\caption{The count of terms in texts of different authors plus the query text.}
\label{table:count}
\end{table}
Next we need to compute the relative frequencies of each term for each author. Then we take the mean and standard deviation of the relative frequencies over
all available authors (Table \ref{table:rel_freq}).
\begin{table}[h]
\center
\begin{tabular}{|l|l|l|l|l|l||l|}
\hline
	 & A1 & A2 & A3 & mean & std & Q\\
\hline
	T1 & 0.085 & 0.075 & 0.021 & 0.061 & 0.028 & 0.179\\
\hline
	T2 & 0.155 & 0.356 & 0.165 & 0.225 & 0.092 & 0.286 \\
\hline
	T3 & 0.386 & 0.356 & 0.086 & 0.276 & 0.135 & 0.321 \\
\hline
	T4 & 0.321 & 0.011 & 0.482 & 0.272 & 0.196 & 0.071 \\
\hline
	T5 & 0.515 & 0.203 & 0.245 & 0.166 & 0.083 & 0.143 \\
\hline
\end{tabular}
\caption{The relative frequency of each term
for each author. Additionaly we computed
the mean and standard deviation over all
authors for each term.}
\label{table:rel_freq}
\end{table}
Next, we compute for every term and every author the $z_{ij}$ score using the formula
\begin{equation}
	z_{ij} = \frac{r_{ij} - \mu_i}{\sigma_i}
\end{equation}
with $r$ as the relative frequency. 

For computing the distance, we just take the
absolute difference between two z scores. The results of this computation are visible in Table \ref{table:abs_diff}
\begin{table}[h]
\center
\begin{tabular}{|l|l|l|l|l|l|}
\hline
	 & A1 & A2 & A3 & Q \\
	 \hline
T1 & 0.895796011645 & 0.499816154218 & -1.39561216586 & 4.203 \\
 \hline 
T2 & -0.765560074165 & 1.41255590087 & -0.646995826702 & 0.654\\
 \hline 
T3 & 0.816843486425 & 0.591365753716 & -1.40820924014 & 0.336\\
 \hline 
T4 & 0.257355759306 & -1.33297269815 & 1.07561693885 & - 1.023\\
 \hline 
T5 & -1.38403154146 & 0.440336373112 & 0.943695168347 & -0.283 \\
\hline
\end{tabular}
\caption{$z_{ij}$ for every author and term, including the query text. }
\label{table:zscores}
\end{table}
and sum over all terms $i$
\begin{equation}
	\Delta(A_j, Q) = \frac{1}{n} \sum_i | z_{ij} - z_{iQ} |
\end{equation}
Summing up the rows of Table \ref{table:zscores} and dividing by the number
of terms, we end up with the following values for all authors available:
\begin{align*}
	 \Delta(A1, Q_1) &= 1.51768031 \\
	 \Delta(A2, Q_1) &= 1.15020531 \\
	 \Delta(A3, Q_1) &= 2.39356869
\end{align*}
So we conclude that A2 is the most probable author in this set of authors.

For the values of exercise b), we do exactly the same computation and end up with the values
\begin{align*}
	 \Delta(A1, Q_2) &= 2.38032735 \\
	 \Delta(A2, Q_2) &= 1.10436749 \\
	 \Delta(A3, Q_2) &= 2.21439687
\end{align*}
and conclude again that A2 as the most probably author.
\begin{table}
\center
\begin{tabular}{|l|l|l|l|l|l|}
\hline
	 & A1 & A2 & A3 \\
	 \hline
T0 & 3.30706320105 & 3.70304305848 & 5.59847137856 \\
 \hline 
T1 & 1.41924001759 & 0.758875957443 & 1.30067577013 \\
 \hline 	
T2 & 0.480993633061 & 0.255515900352 & 1.7440590935 \\
 \hline 
T3 & 1.27987702518 & 0.310451432275 & 2.09813820472 \\
 \hline 
T4 & 1.10122769178 & 0.723140222786 & 1.22649901802 \\
 \hline
\end{tabular}
\caption{Absolute difference to query term.}
\label{table:abs_diff}
\end{table}
\end{document}
