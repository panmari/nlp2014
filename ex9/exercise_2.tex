\documentclass[11pt]{article}
\usepackage{amsmath}
\usepackage{slashbox}
\usepackage{hyperref}

%Gummi|065|=)
\title{\textbf{Answers Exercise 9 - Evaluating Classifications}}
\author{Stefan Moser}
\date{\today}
\begin{document}

\maketitle

\section*{Exercise 2 - Contingency Table}
\begin{table} [h]
\center
\begin{tabular}{c|lll}
	\backslashbox{Term}{Category} & $c_i$ & $c_{-i}$ \\
\hline
	$t_k$ 		& 50 & 80 & 130\\
	$t_{-k}$ 	& 900 & 970 & 2870\\
	 			& 950 & 1050 & 2000\\
\end{tabular}
\caption{Contingency table for this exercise.}
\label{ct}
\end{table}
\subsection*{a) Mutual Information (MI)}
With mutual information (MI) defined as
\begin{equation}
	MI(t_k, c_i) = \log \left( \frac{p(t_k, c_i)}{p(t_k)\cdot p(c_i)} \right)
\end{equation}
According to the article on \url{http://en.wikipedia.org/wiki/Mutual_information}, a good choice
for the base of the logarithm is 2, making the unit of the mutual information \emph{bit}. For this
report I used this base for computing MI. By using the numbers from the contigency table we get to the result
\begin{equation}
	MI(t_k, c_i) = \log \left( \frac{\frac{50}{2000}}{\frac{130}{2000} \cdot \frac{950}{2000}} \right) 
				= \log \left( \frac{50 \cdot 2000}{130 \cdot 950} \right)
				\approx -0.3045 \; .
\end{equation}

\subsection*{b) Odds Ratio (OR)}
With odds ratio (OR) defined as (adapted to the notation of the exercise sheet)
\begin{equation}
	OR(t_k, c_i) = \frac{p(t_k | c_i)}{1 - p(t_k | c_i)} \cdot 
					\frac{1 - p(t_k | c_{-i})}{p(t_k | c_{-i})}
\end{equation}
we again plug in the numbers given from contingency table
\begin{equation}
	OR(t_k, c_i) = \frac{50/950}{1 - 50/950} \cdot 
					\frac{1 - 80/1050}{80/1050} \approx 0.674 \;
\end{equation}

\subsection*{c) Chi-Squared value}
With Chi-Squared value ($\chi^2$) defined as
\begin{equation}
	\chi^2(t_k, c_i) = \frac{|T_r| \cdot \big( p(t_k, c_i) \cdot p(t_{-k}, c_{-i}) -
						 p(t_{k}, c_{-i}) \cdot p(t_{-k}, c_{i}) \big)^2}
						 {p(t_k) \cdot p(t_{-k}) \cdot p(c_{i}) \cdot p(c_{-i})}
\end{equation}
we again plug in the numbers given from contingency table
\begin{equation}
	\chi^2(t_k, c_i) = \frac{2000 \cdot (50 \cdot 970 - 80 \cdot 900)^2}
					{130 \cdot 1870 \cdot 950 \cdot 1050} 
					\approx 4.55
\end{equation}
\subsection*{d) Information Gain (IG)}
With Information Gain (IG) defined as
\begin{equation}
	\chi^2(t_k, c_i) = \frac{|T_r| \cdot \big( p(t_k, c_i) \cdot p(t_{-k}, c_{-i}) -
						 p(t_{k}, c_{-i}) \cdot p(t_{-k}, c_{i}) \big)^2}
						 {p(t_k) \cdot p(t_{-k}) \cdot p(c_{i}) \cdot p(c_{-i})}
\end{equation}
we again plug in the numbers given from contingency table
\begin{equation}
	\chi^2(t_k, c_i) = \frac{2000 \cdot (50 \cdot 970 - 80 \cdot 900)^2}
					{130 \cdot 1870 \cdot 950 \cdot 1050} 
					\approx 4.55
\end{equation}
\end{document}
