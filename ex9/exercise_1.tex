\documentclass[11pt]{article}
\usepackage{amsmath}
\usepackage{slashbox}

%Gummi|065|=)
\title{\textbf{Answers Exercise 9 - Evaluating Classifications}}
\author{Stefan Moser}
\date{\today}
\begin{document}

\maketitle

\section*{Exercise 1 - TF-idf}
The inverse document frequency (idf) is defined as
\begin{equation}
	idf(t) = \log_k \frac{N}{  \max(\{\text{Number of documents containing term } t \}, 1) }
\end{equation}
with $t$ as the evaluated term and $N$ the number of documents in our corpus. The max is
to prevent division by zero.
\subsubsection*{a) Computing idf}
In our case, we have $k=10$ as base of the logarithm.
\begin{equation}
	idf(T_1) = \log_{10} \frac{10000}{100} = \log_{10} 100 = 2
\end{equation}
\begin{equation}
	idf(T_2) = \log_{10} \frac{10000}{200} = \log_{10} 50 \approx 1.699
\end{equation}
\begin{equation}
	idf(T_3) = \log_{10} \frac{10000}{200} = \log_{10} 50 \approx 1.699
\end{equation}
\begin{equation}
	idf(T_4) = \log_{10} \frac{10000}{100} = \log_{10} 100 = 2
\end{equation}
\subsection*{b) Computing tf-idf}
We can compute the $tfidf$ according to the formula
\begin{equation}
		tfidf(T_i, D_j) = tf(T_i, D_j) \cdot idf(T_i) 
\end{equation}
Applied to the first term, we then get
\begin{align}
	tfidf(T_1, D_1) &= tf(T_1, D_1) \cdot idf(T_1) \\
					&= 4 \cdot 2 \\
					&= 8
\end{align}
For all values, we then get
\begin{table}[h]
\begin{tabular}{|l|l|l|l|l|}
\hline
	\backslashbox{Documents}{Terms} & $T_1$ & $T_2$ & $T_3$ & $T_4$\\
\hline
	$D_1$ & $4 \cdot 2 = 8$ & $4 \cdot 1.699 = 6.796$  & $0 \cdot 1.699 = 0$ & $1 \cdot 2 = 2$\\
\hline
	$D_2$ & $4 \cdot 2 = 8$  & $2 \cdot 1.699 = 3.398$ & $10 \cdot 1.699 = 16.99 $ & $5 \cdot 2 = 10$\\
\hline
	$D_2$ & $4 \cdot 2 = 8$  & $2 \cdot 1.699 = 3.398$ & $2 \cdot 1.699 = 3.398$ & $30 \cdot 2 = 60$\\
\hline
\end{tabular}
\caption{Table of tf-idf for every combination of term $T_i$ and document $D_j$}
\end{table}
\subsubsection*{c) Computing query}
The query($T_3, T_4$) can be interpreted as document new document D4 with frequencies
as described in Table \ref{D4}
\begin{table}[h]
\center
\begin{tabular}{|l|l|l|l|l|}
\hline
	Document & $T_1$ & $T_2$ & $T_3$ & $T_4$ \\ 
\hline
	$D_4$ & 1 & 1 & 0 & 0 \\ 
\hline
	$tfidf(D_4)$ & $1 \cdot 2 = 2$ & $1 \cdot 1.699 = 1.699$ & 0 & 0 \\ 
\hline
\end{tabular}
\caption{Frequencies of the query interpreted as document $D_4$ and 
tfidf using the idf computed part a) of this question}
\label{D4}
\end{table}
We need now to compute the most similar document for D4 using the cosine similarity measure
\begin{equation}
	sim(A,B) = \frac{A \cdot B}{|A| \cdot |B|} = \frac{\sum_i A_i \cdot B_i}{\sqrt{\sum_i A_i^2} \cdot \sqrt{\sum_i B_i^2}} \; . 
\end{equation}
\begin{align*}
	sim(D_4, D_1) &= \frac{2 \cdot 8 + 1.699 \cdot 6.796} 
					{\sqrt{2^2 + 1.699^2} \cdot \sqrt{8^2 + 6.796^2 + 0^2 + 2^2}} \approx 0.9823 \\
	sim(D_4, D_2) &= \frac{2 \cdot 8 + 1.699 \cdot 3.398}
					{\sqrt{2^2 + 1.699^2} \cdot \sqrt{8^2 + 3.398^2 + 16.99^2 + 10^2}} \approx 0.3851 \\
	sim(D_4, D_3) &= \frac{2 \cdot 8 + 1.699 \cdot 6.796}
					{\sqrt{2^2 + 1.699^2} \cdot \sqrt{8^2 + 3.398^2 + 3.398^2 + 60^2}} \approx 0.1729
\end{align*}
This leads to the following order in similarity (from most similar to least similar)
\begin{equation}
	D_1 > D_2 > D_3
\end{equation}
\end{document}
