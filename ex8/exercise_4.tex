\documentclass[11pt]{article}
\usepackage{tikz}
\usepackage{tikz-qtree}
%Gummi|065|=)
\title{\textbf{Answers Exercise 8 - Grammar parsing}}
\author{Stefan Moser}
\date{\today}
\begin{document}

\maketitle


\section*{Exercise 4}
Notation is kind of similar as in pyparsing, when there is a + between two grammar parts,
they are actually concatenated
\begin{verbatim}
	plusorminus -> plus | minus
	integer -> plusorminus + number | number
	variable -> plusorminus + variable_name | variable_name
	
	addop -> plus | minus
	multop -> mult | div
	atom -> integer | variable
	
	factor -> atom | atom + factor
	term -> factor | factor + multop + factor
	S -> term | term + addop + term
\end{verbatim}
Terminals
\begin{verbatim}
	number -> 1 | 2 | 3 | 4 | 5 | 6 | 7 | 8 | 9
	plus -> +
	minus -> -
	mult -> *
	div -> /
	variable_name -> {any letter}
\end{verbatim}
This can be more or less directly translated into pyparsing code. 
\end{document}
