\documentclass[11pt]{article}
\usepackage{tikz}
\usepackage{tikz-qtree}
%Gummi|065|=)
\title{\textbf{Answers Exercise 8 - Grammar parsing}}
\author{Stefan Moser}
\date{\today}
\begin{document}

\maketitle

\section*{Exercise 1}
\subsection*{a) "Book that flight"}
There is only one possible tree:

\Tree [.S [.VP [.Verb Book ] [.NP [ [.det that ] [.Nom [.noun flight ] ] ] ] ] ]

\subsection*{b) "Does this flight include a meal?"}
\Tree [.S [.Aux Does ] [.NP [ [.det that ] [.Nom [.noun flight ] ] ] ] [.VP [.Verb include ] 
[ .NP [ .det a ] [ .Nom [.noun meal ] ] ] ] ] 

\section*{Exercise 3}
How to generate L = \{ $a^n b^n | n > 0 $\} with a grammar. If we define our grammar as
\begin{verbatim}
	S -> A
	A -> aAb | ab
\end{verbatim}
So if we want to create the most simple example 'ab', we would go
\begin{verbatim}
	S -> A -> ab
\end{verbatim}
and the more complex one with $n=5$ ('aaaaabbbbb') would be generated as
\begin{verbatim}
	S -> A -> aAb -> aaAbb -> aaaAbbb -> aaaaAbbbb -> aaaaabbbbb
\end{verbatim}

\section*{Exercise 4}
\begin{verbatim}
	S -> Term | Factor | number | variable
	Term -> S + S 
	Factor -> S * S 
\end{verbatim}
Terminals
\begin{verbatim}
	number
	variable
\end{verbatim}
\end{document}
